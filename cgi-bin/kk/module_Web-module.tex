%
% API Documentation for API Documentation
% Module module_Web
%
% Generated by epydoc 3.0.1
% [Tue Jun 12 11:25:41 2012]
%

%%%%%%%%%%%%%%%%%%%%%%%%%%%%%%%%%%%%%%%%%%%%%%%%%%%%%%%%%%%%%%%%%%%%%%%%%%%
%%                          Module Description                           %%
%%%%%%%%%%%%%%%%%%%%%%%%%%%%%%%%%%%%%%%%%%%%%%%%%%%%%%%%%%%%%%%%%%%%%%%%%%%

    \index{module\_Web \textit{(module)}|(}
\section{Module module\_Web}

    \label{module_Web}

Module with classes definitions and methods for creating the Web Page

%%%%%%%%%%%%%%%%%%%%%%%%%%%%%%%%%%%%%%%%%%%%%%%%%%%%%%%%%%%%%%%%%%%%%%%%%%%
%%                               Variables                               %%
%%%%%%%%%%%%%%%%%%%%%%%%%%%%%%%%%%%%%%%%%%%%%%%%%%%%%%%%%%%%%%%%%%%%%%%%%%%

  \subsection{Variables}

    \vspace{-1cm}
\hspace{\varindent}\begin{longtable}{|p{\varnamewidth}|p{\vardescrwidth}|l}
\cline{1-2}
\cline{1-2} \centering \textbf{Name} & \centering \textbf{Description}& \\
\cline{1-2}
\endhead\cline{1-2}\multicolumn{3}{r}{\small\textit{continued on next page}}\\\endfoot\cline{1-2}
\endlastfoot\raggedright Q\-M\- & \raggedright \textbf{Value:} 
{\tt \texttt{'}\texttt{"}\texttt{'}}&\\
\cline{1-2}
\raggedright \_\-\_\-p\-a\-c\-k\-a\-g\-e\-\_\-\_\- & \raggedright \textbf{Value:} 
{\tt None}&\\
\cline{1-2}
\end{longtable}


%%%%%%%%%%%%%%%%%%%%%%%%%%%%%%%%%%%%%%%%%%%%%%%%%%%%%%%%%%%%%%%%%%%%%%%%%%%
%%                           Class Description                           %%
%%%%%%%%%%%%%%%%%%%%%%%%%%%%%%%%%%%%%%%%%%%%%%%%%%%%%%%%%%%%%%%%%%%%%%%%%%%

    \index{module\_Web \textit{(module)}!module\_Web.Image \textit{(class)}|(}
\subsection{Class Image}

    \label{module_Web:Image}

Model for an \emph{image tag} \texttt{<img/>} in HTML language

%%%%%%%%%%%%%%%%%%%%%%%%%%%%%%%%%%%%%%%%%%%%%%%%%%%%%%%%%%%%%%%%%%%%%%%%%%%
%%                                Methods                                %%
%%%%%%%%%%%%%%%%%%%%%%%%%%%%%%%%%%%%%%%%%%%%%%%%%%%%%%%%%%%%%%%%%%%%%%%%%%%

  \subsubsection{Methods}

    \label{module_Web:Image:__init__}
    \index{module\_Web \textit{(module)}!module\_Web.Image \textit{(class)}!module\_Web.Image.\_\_init\_\_ \textit{(method)}}

    \vspace{0.5ex}

\hspace{.8\funcindent}\begin{boxedminipage}{\funcwidth}

    \raggedright \textbf{\_\_init\_\_}(\textit{self}, \textit{fileName}, \textit{directory}={\tt \texttt{'}\texttt{http://127.0.0.1/html5/images/}\texttt{'}}, \textit{classtype}={\tt \texttt{'}\texttt{}\texttt{'}}, \textit{extraparams}={\tt \texttt{'}\texttt{}\texttt{'}})

    \vspace{-1.5ex}

    \rule{\textwidth}{0.5\fboxrule}
\setlength{\parskip}{2ex}

Image class initializer.
%
\begin{quote}

\end{quote}
\setlength{\parskip}{1ex}
      \textbf{Parameters}
      \vspace{-1ex}

      \begin{quote}
        \begin{Ventry}{xxxxxxxxxxx}

          \item[fileName]


:   Line from the file output.txt \emph{(string).}
          \item[directory]


:  absolute path to image's directory \emph{(string).}
          \item[classtype]


:  \emph{(string).}
          \item[extraparams]


:        \emph{(string).}
        \end{Ventry}

      \end{quote}

    \end{boxedminipage}

    \label{module_Web:Image:write}
    \index{module\_Web \textit{(module)}!module\_Web.Image \textit{(class)}!module\_Web.Image.write \textit{(method)}}

    \vspace{0.5ex}

\hspace{.8\funcindent}\begin{boxedminipage}{\funcwidth}

    \raggedright \textbf{write}(\textit{self})

    \vspace{-1.5ex}

    \rule{\textwidth}{0.5\fboxrule}
\setlength{\parskip}{2ex}
\setlength{\parskip}{1ex}
      \textbf{Return Value}
    \vspace{-1ex}

      \begin{quote}

\emph{string} with the HTML code for the \textbf{Image structure} for being used in other structures.
      \end{quote}

    \end{boxedminipage}

    \index{module\_Web \textit{(module)}!module\_Web.Image \textit{(class)}|)}

%%%%%%%%%%%%%%%%%%%%%%%%%%%%%%%%%%%%%%%%%%%%%%%%%%%%%%%%%%%%%%%%%%%%%%%%%%%
%%                           Class Description                           %%
%%%%%%%%%%%%%%%%%%%%%%%%%%%%%%%%%%%%%%%%%%%%%%%%%%%%%%%%%%%%%%%%%%%%%%%%%%%

    \index{module\_Web \textit{(module)}!module\_Web.Input \textit{(class)}|(}
\subsection{Class Input}

    \label{module_Web:Input}

Model for an \emph{input tag} \texttt{<input/>} in HTML language

%%%%%%%%%%%%%%%%%%%%%%%%%%%%%%%%%%%%%%%%%%%%%%%%%%%%%%%%%%%%%%%%%%%%%%%%%%%
%%                                Methods                                %%
%%%%%%%%%%%%%%%%%%%%%%%%%%%%%%%%%%%%%%%%%%%%%%%%%%%%%%%%%%%%%%%%%%%%%%%%%%%

  \subsubsection{Methods}

    \label{module_Web:Input:__init__}
    \index{module\_Web \textit{(module)}!module\_Web.Input \textit{(class)}!module\_Web.Input.\_\_init\_\_ \textit{(method)}}

    \vspace{0.5ex}

\hspace{.8\funcindent}\begin{boxedminipage}{\funcwidth}

    \raggedright \textbf{\_\_init\_\_}(\textit{self}, \textit{\_type}, \textit{\_id}={\tt \texttt{'}\texttt{}\texttt{'}}, \textit{name}={\tt \texttt{'}\texttt{}\texttt{'}}, \textit{value}={\tt \texttt{'}\texttt{}\texttt{'}}, \textit{classtype}={\tt \texttt{'}\texttt{}\texttt{'}}, \textit{event}={\tt \texttt{'}\texttt{}\texttt{'}})

    \vspace{-1.5ex}

    \rule{\textwidth}{0.5\fboxrule}
\setlength{\parskip}{2ex}

Input class initializer.
%
\begin{quote}

\end{quote}
\setlength{\parskip}{1ex}
      \textbf{Parameters}
      \vspace{-1ex}

      \begin{quote}
        \begin{Ventry}{xxxxxxxxx}

          \item[\_type]


:              Type of input: ``file'',``submit'' or ``button'' \emph{(string).}
          \item[\_id]


:                Identifier \emph{(string).}
          \item[name]


:               \emph{(string).}
          \item[value]


:              \emph{(string).}
          \item[classtype]


:  \emph{(string).}
          \item[event]


:              Describes what to do when a specif event take place: ``onclick'' \emph{(string).}
        \end{Ventry}

      \end{quote}

      \textbf{Raises}
    \vspace{-1ex}

      \begin{quote}
        \begin{description}

          \item[\texttt{ArgumentError}]


: Raised when \emph{\_type} is not on its list values.
        \end{description}

      \end{quote}

    \end{boxedminipage}

    \label{module_Web:Input:write}
    \index{module\_Web \textit{(module)}!module\_Web.Input \textit{(class)}!module\_Web.Input.write \textit{(method)}}

    \vspace{0.5ex}

\hspace{.8\funcindent}\begin{boxedminipage}{\funcwidth}

    \raggedright \textbf{write}(\textit{self})

    \vspace{-1.5ex}

    \rule{\textwidth}{0.5\fboxrule}
\setlength{\parskip}{2ex}
\setlength{\parskip}{1ex}
      \textbf{Return Value}
    \vspace{-1ex}

      \begin{quote}

\emph{string} with the HTML code for the \textbf{Input structure} for being used in other structures.
      \end{quote}

    \end{boxedminipage}

    \index{module\_Web \textit{(module)}!module\_Web.Input \textit{(class)}|)}

%%%%%%%%%%%%%%%%%%%%%%%%%%%%%%%%%%%%%%%%%%%%%%%%%%%%%%%%%%%%%%%%%%%%%%%%%%%
%%                           Class Description                           %%
%%%%%%%%%%%%%%%%%%%%%%%%%%%%%%%%%%%%%%%%%%%%%%%%%%%%%%%%%%%%%%%%%%%%%%%%%%%

    \index{module\_Web \textit{(module)}!module\_Web.Form \textit{(class)}|(}
\subsection{Class Form}

    \label{module_Web:Form}

Model for a \emph{form tag} \texttt{<form></form>} in HTML language

%%%%%%%%%%%%%%%%%%%%%%%%%%%%%%%%%%%%%%%%%%%%%%%%%%%%%%%%%%%%%%%%%%%%%%%%%%%
%%                                Methods                                %%
%%%%%%%%%%%%%%%%%%%%%%%%%%%%%%%%%%%%%%%%%%%%%%%%%%%%%%%%%%%%%%%%%%%%%%%%%%%

  \subsubsection{Methods}

    \label{module_Web:Form:__init__}
    \index{module\_Web \textit{(module)}!module\_Web.Form \textit{(class)}!module\_Web.Form.\_\_init\_\_ \textit{(method)}}

    \vspace{0.5ex}

\hspace{.8\funcindent}\begin{boxedminipage}{\funcwidth}

    \raggedright \textbf{\_\_init\_\_}(\textit{self}, \textit{content}, \textit{action}, \textit{enctype}={\tt \texttt{'}\texttt{multipart/form-data}\texttt{'}}, \textit{method}={\tt \texttt{'}\texttt{post}\texttt{'}}, \textit{\_id}={\tt \texttt{'}\texttt{}\texttt{'}}, \textit{classtype}={\tt \texttt{'}\texttt{}\texttt{'}})

    \vspace{-1.5ex}

    \rule{\textwidth}{0.5\fboxrule}
\setlength{\parskip}{2ex}

Form class initializer.
%
\begin{quote}

\end{quote}
\setlength{\parskip}{1ex}
      \textbf{Parameters}
      \vspace{-1ex}

      \begin{quote}
        \begin{Ventry}{xxxxxxxxx}

          \item[content]


:    Content of the form. Usually with text and input elements inside. \emph{(string).}
          \item[action]


:             Indicates the URL which processes the form data. \emph{(string).}
          \item[enctype]


:    Type of codification: ``multipart/form-data'' or ``application/x-www-form-urlencoded''. \emph{(string).}
          \item[method]


:             HTTP method for sending the form: ``post'' or ``get''. \emph{(string).}
          \item[\_id]


:                Identifier \emph{(string).}
          \item[classtype]


:  \emph{(string).}
        \end{Ventry}

      \end{quote}

      \textbf{Raises}
    \vspace{-1ex}

      \begin{quote}
        \begin{description}

          \item[\texttt{ArgumentError}]


: Raised when \emph{enctype} or \emph{method} is not on their list values.
        \end{description}

      \end{quote}

    \end{boxedminipage}

    \label{module_Web:Form:write}
    \index{module\_Web \textit{(module)}!module\_Web.Form \textit{(class)}!module\_Web.Form.write \textit{(method)}}

    \vspace{0.5ex}

\hspace{.8\funcindent}\begin{boxedminipage}{\funcwidth}

    \raggedright \textbf{write}(\textit{self})

    \vspace{-1.5ex}

    \rule{\textwidth}{0.5\fboxrule}
\setlength{\parskip}{2ex}
\setlength{\parskip}{1ex}
      \textbf{Return Value}
    \vspace{-1ex}

      \begin{quote}

\emph{string} with the HTML code for the \textbf{Form structure} for being used in other structures.
      \end{quote}

    \end{boxedminipage}

    \index{module\_Web \textit{(module)}!module\_Web.Form \textit{(class)}|)}

%%%%%%%%%%%%%%%%%%%%%%%%%%%%%%%%%%%%%%%%%%%%%%%%%%%%%%%%%%%%%%%%%%%%%%%%%%%
%%                           Class Description                           %%
%%%%%%%%%%%%%%%%%%%%%%%%%%%%%%%%%%%%%%%%%%%%%%%%%%%%%%%%%%%%%%%%%%%%%%%%%%%

    \index{module\_Web \textit{(module)}!module\_Web.DIV \textit{(class)}|(}
\subsection{Class DIV}

    \label{module_Web:DIV}

Model for a \emph{div tag} \texttt{<div></div>} in HTML language

%%%%%%%%%%%%%%%%%%%%%%%%%%%%%%%%%%%%%%%%%%%%%%%%%%%%%%%%%%%%%%%%%%%%%%%%%%%
%%                                Methods                                %%
%%%%%%%%%%%%%%%%%%%%%%%%%%%%%%%%%%%%%%%%%%%%%%%%%%%%%%%%%%%%%%%%%%%%%%%%%%%

  \subsubsection{Methods}

    \label{module_Web:DIV:__init__}
    \index{module\_Web \textit{(module)}!module\_Web.DIV \textit{(class)}!module\_Web.DIV.\_\_init\_\_ \textit{(method)}}

    \vspace{0.5ex}

\hspace{.8\funcindent}\begin{boxedminipage}{\funcwidth}

    \raggedright \textbf{\_\_init\_\_}(\textit{self}, \textit{content}, \textit{\_id}={\tt \texttt{'}\texttt{}\texttt{'}}, \textit{classtype}={\tt \texttt{'}\texttt{}\texttt{'}})

    \vspace{-1.5ex}

    \rule{\textwidth}{0.5\fboxrule}
\setlength{\parskip}{2ex}

Form class initializer.
%
\begin{quote}

\end{quote}
\setlength{\parskip}{1ex}
      \textbf{Parameters}
      \vspace{-1ex}

      \begin{quote}
        \begin{Ventry}{xxxxxxxxx}

          \item[content]


:    Content of the form. Usually with text and tags elements inside. \emph{(string).}
          \item[\_id]


:                Identifier \emph{(string).}
          \item[classtype]


:  \emph{(string).}
        \end{Ventry}

      \end{quote}

    \end{boxedminipage}

    \label{module_Web:DIV:write}
    \index{module\_Web \textit{(module)}!module\_Web.DIV \textit{(class)}!module\_Web.DIV.write \textit{(method)}}

    \vspace{0.5ex}

\hspace{.8\funcindent}\begin{boxedminipage}{\funcwidth}

    \raggedright \textbf{write}(\textit{self})

    \vspace{-1.5ex}

    \rule{\textwidth}{0.5\fboxrule}
\setlength{\parskip}{2ex}
\setlength{\parskip}{1ex}
      \textbf{Return Value}
    \vspace{-1ex}

      \begin{quote}

\emph{string} with the HTML code for the \textbf{DIV structure} for being used in other structures.
      \end{quote}

    \end{boxedminipage}

    \index{module\_Web \textit{(module)}!module\_Web.DIV \textit{(class)}|)}

%%%%%%%%%%%%%%%%%%%%%%%%%%%%%%%%%%%%%%%%%%%%%%%%%%%%%%%%%%%%%%%%%%%%%%%%%%%
%%                           Class Description                           %%
%%%%%%%%%%%%%%%%%%%%%%%%%%%%%%%%%%%%%%%%%%%%%%%%%%%%%%%%%%%%%%%%%%%%%%%%%%%

    \index{module\_Web \textit{(module)}!module\_Web.HTML \textit{(class)}|(}
\subsection{Class HTML}

    \label{module_Web:HTML}

Model for a \emph{htmltag} \texttt{<html></html>} in HTML language

%%%%%%%%%%%%%%%%%%%%%%%%%%%%%%%%%%%%%%%%%%%%%%%%%%%%%%%%%%%%%%%%%%%%%%%%%%%
%%                                Methods                                %%
%%%%%%%%%%%%%%%%%%%%%%%%%%%%%%%%%%%%%%%%%%%%%%%%%%%%%%%%%%%%%%%%%%%%%%%%%%%

  \subsubsection{Methods}

    \label{module_Web:HTML:__init__}
    \index{module\_Web \textit{(module)}!module\_Web.HTML \textit{(class)}!module\_Web.HTML.\_\_init\_\_ \textit{(method)}}

    \vspace{0.5ex}

\hspace{.8\funcindent}\begin{boxedminipage}{\funcwidth}

    \raggedright \textbf{\_\_init\_\_}(\textit{self}, \textit{title\_TAB}={\tt \texttt{'}\texttt{}\texttt{'}}, \textit{style\_Files}={\tt \texttt{[}\texttt{]}}, \textit{script\_Files}={\tt \texttt{[}\texttt{]}}, \textit{header}={\tt \texttt{'}\texttt{}\texttt{'}}, \textit{html\_Body}={\tt \texttt{'}\texttt{}\texttt{'}}, \textit{footer}={\tt \texttt{'}\texttt{}\texttt{'}})

    \vspace{-1.5ex}

    \rule{\textwidth}{0.5\fboxrule}
\setlength{\parskip}{2ex}

Form class initializer.
%
\begin{quote}

\end{quote}
\setlength{\parskip}{1ex}
      \textbf{Parameters}
      \vspace{-1ex}

      \begin{quote}
        \begin{Ventry}{xxxxxxxxxxxx}

          \item[title\_TAB]


:                          This text will be displayed in the tab browser. \emph{(string).}
          \item[style\_Files]


:                Stores the names of the .css files \emph{(string).}
          \item[script\_Files]


:               Stores the names of the .js (javascript) files \emph{(string).}
          \item[header]


:                             The main and bigger text of the website. \emph{(string).}
          \item[html\_Body]


:                  \emph{(string).}
          \item[footer]


:                             \emph{(string).}
        \end{Ventry}

      \end{quote}

    \end{boxedminipage}

    \label{module_Web:HTML:addTitle}
    \index{module\_Web \textit{(module)}!module\_Web.HTML \textit{(class)}!module\_Web.HTML.addTitle \textit{(method)}}

    \vspace{0.5ex}

\hspace{.8\funcindent}\begin{boxedminipage}{\funcwidth}

    \raggedright \textbf{addTitle}(\textit{self}, \textit{title\_TAB})

\setlength{\parskip}{2ex}
\setlength{\parskip}{1ex}
    \end{boxedminipage}

    \label{module_Web:HTML:add_styleFiles}
    \index{module\_Web \textit{(module)}!module\_Web.HTML \textit{(class)}!module\_Web.HTML.add\_styleFiles \textit{(method)}}

    \vspace{0.5ex}

\hspace{.8\funcindent}\begin{boxedminipage}{\funcwidth}

    \raggedright \textbf{add\_styleFiles}(\textit{self}, \textit{fileName})

\setlength{\parskip}{2ex}
\setlength{\parskip}{1ex}
    \end{boxedminipage}

    \label{module_Web:HTML:add_scriptFiles}
    \index{module\_Web \textit{(module)}!module\_Web.HTML \textit{(class)}!module\_Web.HTML.add\_scriptFiles \textit{(method)}}

    \vspace{0.5ex}

\hspace{.8\funcindent}\begin{boxedminipage}{\funcwidth}

    \raggedright \textbf{add\_scriptFiles}(\textit{self}, \textit{fileName})

\setlength{\parskip}{2ex}
\setlength{\parskip}{1ex}
    \end{boxedminipage}

    \label{module_Web:HTML:addHeader}
    \index{module\_Web \textit{(module)}!module\_Web.HTML \textit{(class)}!module\_Web.HTML.addHeader \textit{(method)}}

    \vspace{0.5ex}

\hspace{.8\funcindent}\begin{boxedminipage}{\funcwidth}

    \raggedright \textbf{addHeader}(\textit{self}, \textit{header})

\setlength{\parskip}{2ex}
\setlength{\parskip}{1ex}
    \end{boxedminipage}

    \label{module_Web:HTML:addBody}
    \index{module\_Web \textit{(module)}!module\_Web.HTML \textit{(class)}!module\_Web.HTML.addBody \textit{(method)}}

    \vspace{0.5ex}

\hspace{.8\funcindent}\begin{boxedminipage}{\funcwidth}

    \raggedright \textbf{addBody}(\textit{self}, \textit{content})

\setlength{\parskip}{2ex}
\setlength{\parskip}{1ex}
    \end{boxedminipage}

    \label{module_Web:HTML:addFooter}
    \index{module\_Web \textit{(module)}!module\_Web.HTML \textit{(class)}!module\_Web.HTML.addFooter \textit{(method)}}

    \vspace{0.5ex}

\hspace{.8\funcindent}\begin{boxedminipage}{\funcwidth}

    \raggedright \textbf{addFooter}(\textit{self}, \textit{footer})

\setlength{\parskip}{2ex}
\setlength{\parskip}{1ex}
    \end{boxedminipage}

    \label{module_Web:HTML:__str__}
    \index{module\_Web \textit{(module)}!module\_Web.HTML \textit{(class)}!module\_Web.HTML.\_\_str\_\_ \textit{(method)}}

    \vspace{0.5ex}

\hspace{.8\funcindent}\begin{boxedminipage}{\funcwidth}

    \raggedright \textbf{\_\_str\_\_}(\textit{self})

    \vspace{-1.5ex}

    \rule{\textwidth}{0.5\fboxrule}
\setlength{\parskip}{2ex}
\setlength{\parskip}{1ex}
      \textbf{Return Value}
    \vspace{-1ex}

      \begin{quote}

\emph{string} with the HTML code for the \textbf{HTML structure}.
      \end{quote}

    \end{boxedminipage}

    \label{module_Web:HTML:WriteHTMLfile}
    \index{module\_Web \textit{(module)}!module\_Web.HTML \textit{(class)}!module\_Web.HTML.WriteHTMLfile \textit{(method)}}

    \vspace{0.5ex}

\hspace{.8\funcindent}\begin{boxedminipage}{\funcwidth}

    \raggedright \textbf{WriteHTMLfile}(\textit{self}, \textit{fileName})

    \vspace{-1.5ex}

    \rule{\textwidth}{0.5\fboxrule}
\setlength{\parskip}{2ex}

Prints the html code in an output file.
\setlength{\parskip}{1ex}
    \end{boxedminipage}

    \index{module\_Web \textit{(module)}!module\_Web.HTML \textit{(class)}|)}
    \index{module\_Web \textit{(module)}|)}
